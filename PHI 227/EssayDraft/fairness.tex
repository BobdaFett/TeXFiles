\section{Fairness}
\subsection{What is fairness?}
Fairness, as a concept, is quite hard to define. There's multiple different definitions that have
completely different applications depending on what you're applying it to. There's also what I
would consider to be "levels" of fairness. These "levels" are created as a direct result of the
fact that, in some cases, one can never be "completely" fair. A couple of the examples that we've
discussed in class are closely related to the distribution of resources, especially in a capitalistic
society, where your gains are almost directly related to how much work you've put in. There's always
been a logistical and ethical debate, especially in politics, centered around the distribution of
wealth in a society that people argue is, by definition, unfair. On top of this, there's the
implicit discrimination of black people in our society, creating an even larger divide. When we
attempt to relate this to AI, there are a couple similarities. Just like the arguments involving
people of color in our society, access to AI is incredibly discriminatory. Not everyone in the 
world is able to access AI, whether for financial reasons, racial reasons, etc. Not everyone has
access to the technology prerequisites in order to understand how it works, or to be able to get
to the website that hosts our interfaces into various models. Past this, you can enter into the
realm of competition with the AI corporations - there's little to no way to compete with these
gigantic companies. Since the power of an AI is directy linked to how much data it has access to
and how much processing power is behind it, someone starting a new company has almost no way to
compete with the corporations, simply due to the fact that they do not have access to the base
requirements of the AI's processing power. This could be seen as a violation of fairness.

\subsection{Example of fairness in AI}
A direct example of the violation of this principle is in a hypothetical situation. An age old
debate in the world of politics, even on the local level, is the debate of whether or not to
create some sort of predictive policing system. This system would use AI to predict where and
when crimes would occur, and then send police to those areas. The algorithm would also be able
to direct police to specific people, based on their likelihood to commit a crime. Besides the 
obvious implications of overarching govenmental power and the potential for abuse, there's also
the issue of fairness, especially in the case of impoverished communities and individuals. This
is a direct violation of the principle of fairness, as it would be targeting a specific group
of people, and would be using AI to do so. These kinds of systems are designed, in theory, to
target the people that are most likely to commit a crime, but in reality, they target the people
that are most likely to be arrested. This also creates something called a feedback loop, where 
the people that are arrested are then more likely to be arrested again, and so on. These algorithms
would also need to breach the privacy of the people that they're targeting, which is another
issue that this presents. On top of this, there's also the issue of accusing someone of a crime
that they haven't committed yet. George Orwell's 1984 is a great example of this, where the
government is able to arrest people for thought crimes, or crimes that they haven't committed
yet, but are "likely" to commit. All of these together create a system that is inherently unfair
to the people that it's targeting.


