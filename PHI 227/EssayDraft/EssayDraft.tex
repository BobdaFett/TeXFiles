\documentclass[12pt]{article}
\usepackage[letterpaper, margin=1in]{geometry}
\usepackage[backend=biber, style=apa, citestyle=numeric]{biblatex}
\usepackage{fancyhdr}
\usepackage[skip=12pt plus1pt]{parskip}
\usepackage{setspace}
\usepackage{titling}

\pagestyle{fancy}
\fancyhead{}
\fancyhead[LO]{Lucas Vas}
\fancyhead[RO]{ESSAY DRAFT}
\fancyhead[CO]{\thepage}

\fancypagestyle{firstPage}{
    \fancyhf{}
    \fancyfoot[C]{\thepage}
}

\addbibresource{sources.bib}

\title{Essay Draft}
\author{Lucas Vas}
\date{}

\setlength{\parindent}{0pt}
\setlength{\headheight}{15pt}

% TODO - Add sources into the sections of the paper. This should raise the word count to the limit.

\doublespacing
\begin{document}
    
    \thispagestyle{firstPage}
    \maketitle
    
    AI has made leaps and bounds over the past several years. It's starting to become a part of
    life, something that we all should understand how to use, at least on a very basic level. Its
    capabilities start at something as simple as summarizing a Google search, to writing code
    for a program or webpage that looks good and functions (relatively) well. It can "understand"
    information that's fed to it from every field - the medical field, the math field, sports, 
    judicial branch - if there's text to read, then it can interpret it in some way. AI is quite
    the tool, by far one of the most industry-shaking technological advancements in the 21st century,
    but it still has its issues. One of the largest issues with AI, in its current state, is that
    no one really understands how it works. Not one person can trace the decision-making process
    back to something we can understand as humans, however we still allow the AI to influence our
    own creative process and understanding of the world. I believe that the lack of explainability
    within AI models, with a focus on large language models that have been used to shape our
    decisions, raises serious ethical concerns due to the lack of fairness, accountability, and 
    transparency, which leads to a lack of trust and holds back fully responsible deployment of AI
    into our everyday lives.

    \include{Transparency}
    \include{Accountability}
    \include{Fairness}

    \section{Conclusion}
    Overall, AI is a technology that is still in its infancy. It's still being developed and
    improved upon. It's shaking up the world of technology, and it's changing the way that we think
    about the world. However, it's also creating a lot of problems that we need to address. We need
    to be able to hold AI accountable for its actions, and we need to be able to explain how it works,
    in a way that everyone can understand. We need to be able to make sure that AI is fair, and that
    it's not being used to target specific groups of people. In general, we need to be able to ensure
    that AI is being used responsibly, and that it will not and cannot be used to harm people. For
    right now, AI is a fun toy that we can play with, but should not be used for anything serious,
    especially to make decisions that are possibly life changing.

\end{document}
