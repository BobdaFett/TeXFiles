\documentclass{article}
\usepackage[letterpaper, margin=.75in]{geometry}

\begin{document}

\author{Lucas Vas and Autumn Perlman}
\title{Separation of a Ternary Mixture}
\date{09/10/2024}

\maketitle

  \section*{Data and Calculations:}
  \begin{table}[h!]
    \centering
    \begin{tabular}{|| c c c c ||}
      \hline
      Cont. Name & Mass of Container (g) & Mass w/ Substance (g) & Mass of Substance (g) \\
      \hline\hline
      Beaker 1 & 89.23 & 90.80 & 1.57 \\
      \hline
      Beaker 2 & 96.27 & 96.96 & 0.69 \\
      \hline
      Evap. Dish & 54.03 & 54.33 & 0.30 \\
      \hline
      Watch Glass/Filter & 45.07 & 45.47 & 0.40 \\
      \hline
    \end{tabular}
    \label{table:1}
    \caption{Masses of all equipment and samples.}
  \end{table}
  \begin{center}
    Mass of substance in Beaker 1 $ = 90.80 - 89.23 = 1.57$g
  \end{center}

  \begin{table}[h!]
    \centering
    \begin{tabular}{|| c c c c ||}
      \hline
      Substance & Expected Mass (g) & Recovered Mass (g) & \% Recovered Mass \\
      \hline\hline
      Sample & 1.57 & 1.39 & 88.5 \\
      \hline
      NaCl & 0.62 & 0.69 & 49.6 \\
      \hline
      SiO\textsubscript{2} & 0.31 & 0.30 & 21.5 \\
      \hline
      CaCO\textsubscript{3} & 0.62 & 0.40 & 28.7 \\
      \hline
    \end{tabular}
    \label{table:2}
    \caption{Masses of recovered samples}
  \end{table}
  \begin{center}
    Percent Error of Recovered Sample $= \frac{1.57 - 1.39}{1.57} * 100 = 88.5\%$
  \end{center}

  \section*{Results and Discussion:}

  Results in this lab allowed for the mixture to be separated into three different materials.
  NaCl (salt), CaCO\textsubscript{3} (chalk), and SiO\textsubscript{2} (sand) were succesfully separated into three groups and
  weighed accordingly. In the original mixture, 20\% of the mixture was composed of sand, and then 40\%
  each was salt and chalk. When the materials were recovered, 21.5\% was sand, 28.7\% was chalk, and
  49.6\% was salt. 0.18g of the original sample was lost, for a total of 1.39g recovered.

  Some of the sources of error in this experiment may have been due to overboiling of the mixture,
  which may have happened a couple times throughout. This would have resulted in losing some material,
  whether it was due to splashing or evaporation of said substance. A second source of error may have
  been from transferring the substance into different containers - i.e., pouring the mixture into a
  different container and being unable to wash all remnants out of the original container. This would
  mean that the substances were stuck to the inside of the original container, and were then discarded
  in order to wash the container for reuse.

  Some possible improvements would be extra care while transporting the mixture, and simply paying attention
  to the mixture while it is boiling and/or lowering the temperature of the hot plate.

\end{document}
