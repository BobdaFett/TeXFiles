\documentclass[12pt]{article}
\usepackage{rocca-homework}
\usepackage[version=4]{mhchem}
\usepackage{tabularx}

\title{Determination of a Chemical Formula}
\author{Lucas Vas and Autumn Perlman}
\date{October 1, 2024}

\begin{document}

\maketitle

\section*{Purpose}

The purpose of this experiment was to determine the chemical formula of a known copper
chloride hydrate compound.

\section*{Theory}

A hydrate is a compound that has water molecules directly integrated into the crystal
structure. This means that the water molecules are not just physically attached to the
compound, but are actually part of the compound. Hydrates must be specified with a
coefficient that indicates the number of water molecules per formula unit. For example,
copper(II) chloride dihydrate is written as \ce{CuCl2 * 2H2O}, which indicates that there
are two water molecules for every formula unit of copper(II) chloride.

Subscripts in a chemical formula are used to indicate the number of atoms of each element
in a formula unit. For example, the formula \ce{CuCl_2} indicates that there is one copper
atom and two chlorine atoms in a formula unit of copper(II) chloride. The coefficient in
the hydrate mentioned above means that there are two water molecules for every formula unit
of copper(II) chloride.

Decomposition analysis is a technique used to determine the chemical formula of a compound.
In this specific experiment, the compound was heated to evaporate the water molecules, thus
leaving only the anhydrous compound. This means that the mass of the anhydrous compound can
be measured and used throughout the rest of the experiment.

The general formula for the decomposition of a hydrate is as follows, where $M$ and $N$ are
placeholder elements:
\begin{equation*}
    \ce{M_xN_y * 2H2O -> M_xN_y + 2H2O}
\end{equation*}

We removed copper from the formula by reacting the copper with an aluminum wire. The general
reaction formula is as follows:
\begin{equation*}
    \ce{Cu_yCl_z + Al_x -> Al_xCl_z + \mathit{y}Cu}
\end{equation*}

In the end, the amount of chlorine in the compound was measured by subtraction. Since the mass
of the full compound was measured, and then the mass of just the copper was measured, the mass
of the chlorine could be calculated by subtracting the mass of the copper from the mass of the
full compound.

\section*{Procedure}

This lab procedure was followed as written in the Determination of a Chemical Formula lab
manual. No changes were made.

\section*{Data and Calculations}

\begin{table}[h!]
    \centering
    \begin{tabularx}{0.8\textwidth}{| c >{\raggedright\arraybackslash}X | c |}
        \hline
        \multicolumn{2}{|c|}{\textbf{Object}} & \textbf{Mass (g)} \\
        \hline \hline
        \multicolumn{2}{|c|}{Crucible} & 7.881 \\
                & w/ Hydrated Sample & 8.891 \\
                & w/ Dehydrated Sample & 8.691 \\
        \hline
        \multicolumn{2}{|c|}{Filter Paper and Watch Glass} & 44.863 \\
            & w/ Copper & 45.208 \\
        \hline
    \end{tabularx}
    \label{table:1}
    \caption{Mass of objects used in the experiment}
\end{table}

\begin{equation*}
    \begin{split}
        \text{\textbf{Mass of hydrated sample}} & = \text{Mass of Sample and Crucible} - \text{Mass of Crucible} \\
        & = 8.891 - 7.881 \\
        & = 0.345
    \end{split}
\end{equation*}

\begin{table}[h!]
    \centering
    \begin{tabular}{| c | c c |}
        \hline
        \textbf{Substance} & \textbf{Mass (g)} & \textbf{Moles} \\
        \hline \hline
        Copper & 0.345 & 0.005249 \\
        Chlorine & 0.465 & 0.013115 \\
        Water & 0.200 & 0.011098 \\
        \hline
    \end{tabular}
    \label{table:2}
    \caption{Observed masses of substances}
\end{table}

\begin{equation*}
    \begin{split}
        \text{\textbf{Moles of copper}} & = \frac{\text{Mass of copper}}{\text{Molar mass of copper}} \\
        & = \frac{0.345 \text{ g}}{63.54 \text{ g/mol}} \\
        & = 0.005249 \text{ mol} \\
        \hline
        \text{\textbf{Ratio of copper to chlorine}} & = \left(\frac{\text{moles Cl}}{\text{moles Cu}}\right) \\
        & = \left(\frac{0.013115 \text{ mol}}{0.005249 \text{ mol}}\right) \\
        & \approx 2.5 \\
        & \text{\textit{This number is rounded down to a whole number.}}
    \end{split}
\end{equation*}

Formula of dehydrated sample (from ratios) = \ce{CuCl2}

Formula of hydrated sample (from ratios) = \ce{CuCl2 * 2H2O}

\clearpage

\section*{Results and Discussion}

The formula of the hydrated sample was determined to be \ce{CuCl2 * H20}. This was found by
using the mass of the hydrated sample and the mass of the dehydrated sample to determine the
mass of the water in the hydrated sample. The mass of the water was then used to determine the
number of moles of water in the hydrated sample. The number of moles of water was then used to
determine the number of moles of copper and chlorine in the hydrated sample. Finally, the ratio
of copper to chlorine was then used to determine the formula of the hydrated sample.

The first reaction that takes place is simple dehydration. The mixture was heated with a Bunsen
burner to evaporate the water molecules from the compound. While this was happening, there was
some steam released from the compound, which was the water evaporating. The second reaction was
the reaction of the copper with the aluminum wire. This reaction generated some heat and a small
amount of \ce{H2} gas. This drew out the copper from the compound, leaving only the chlorine and
the aluminum in the beaker. The results of this reaction were poured into a filter paper and the
copper was separated from the chlorine and aluminum. This was dried and weighed. The mass of the
chlorine was calculated by subtraction.

The formula that was created before the experiment was:
\begin{equation*}
    \ce{Cu_xCl_y * \text{\textit{Z}}H2O -> Cu_xCl_y + \text{\textit{Z}}H2O}
\end{equation*}
This matches with the final equation, which resulted in the formula of the hydrated sample being
\ce{CuCl2 * 2H2O}. The formula of the dehydrated sample was found to be \ce{CuCl2}.

The results of this experiment were consistent with the expected results. Masses were measured and
calculated, and the determined formula of the hydrated sample matched the true formula given by the
lab professor. However, some sources of error may be that the copper was not completely removed from
the beaker and/or the compound itself, or that the chlorine was not completely separated from the
copper. This could have led to slight inaccuracies in the final results, however, since the results
were consistent with the expected results, these errors were likely minimal.

\section*{Conclusion}

The true formula of the reactant compound was determined to be \ce{CuCl2 * 2H2O}, and this was consistent
with the expected results. Therefore, it is concluded that the experiment performed was successful, with
minimal errors.

\section*{References}

[1] Unknown Author. (Unknown Date). \textit{Determination of a Chemical Formula}. Lab Handout.

\end{document}
