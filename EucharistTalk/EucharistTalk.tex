\documentclass[12pt]{article}

\usepackage[letterpaper, margin=1in]{geometry}
\usepackage{palatino}
\usepackage[skip=12pt plus1pt]{parskip}
\usepackage{setspace}

\setlength{\parindent}{0pt}

\title{Eucharist Talk}
\author{Lucas Vas}
\date{\today}

\begin{document}

  \maketitle
  \clearpage

    Hi all! My name is Lucas, and I'm another Vas (yes, my dad is here). I'm currently a computer science student
    at WCSU, and I'm almost 22 years old - give it a couple days.

    Last year, I came to this retreat and it was amazing! I met quite a few people, all of which have followed up
    with me (or me with them) at some point within that year. (people maybe here) Of course, I'm sure you've heard
    these same sorts of experiences from other people - if nothing else, the people that have given their own talks
    before me today and yesterday are all former retreat-goers. Personally, I think that really says something about
    this retreat and how it works. Take advantage of that!

    I was asked to give this talk about the Eucharist. Of course, that's something that's integral to what makes
    anyone Catholic. The Eucharist is a very special thing - of course it looks like a piece of bread to us all
    the time, but it's something different. It's different enough that it's got it's own liturgy in the mass - literally
    a whole part of the mass completely dedicated to the worship and consumption of this sacrament - every single mass.
    I'd say that this should point something out as being super important.

    If you're anything like me, though, it really doesn't even matter. You could dedicate the whole mass to it rather
    than just a chunk. I've had tons of people come up here and tell me about how it's amazing and how it will change
    your life. Sometimes I think I've heard everything about it that I possibly could have, and everything's started
    feeling pretty repetitive. ``This is the body of Christ!" Yeah, alright, I've heard that one. ``This little
    piece of bread has completely changed my life!" Yep, that's another one that I've already heard too.

    In my opinion, hearing about these things doesn't always make it easier to believe in. Even though I'm fairly young in the grand
    scale of things, I think I'm able to say that this isn't new information. Having someone come up to you and say
    ``They're giving out free stuff over there!" doesn't necessarily mean that you believe them. In fact, you're
    probably pretty skeptical about that statement because it tends to lead to something that's... we'll say not so
    great. Think some random guy with a white van kind of not great. It's just not something that we believe until
    we actually see the sign outside the store that says ``free stuff." On top of this sort of skepticism, I'm someone
    that's really likes to have physical proof in front of me, something that I can see, touch, taste, smell, and/or
    hear. Sometimes, really all the time, the Eucharist is really good at hiding all of that. If that's the body and
    blood of Christ, why can't I see the skin? Why can't I taste it?

    When I was asked to give this talk, I thought it was an interesting turn of events. Of all the people that could
    be asked to write this, it's going to be the one person on that last retreat that has a problem with believing
    in the thing I'm supposed to be writing about? This sounds great. What could go wrong? Not only that, but I'm also
    supposed to show off the healing power of this sacrament too?

    Either way, back in high school (4 \textit{long} years ago) I was someone that I'd say was different. I was obviously
    just a little less mature, didn't necessarily think out all of my decisions too far, and I was horrible at doing homework.
    I mean, I'm still pretty bad at doing homework, but that's besides the point. At some point, I got a girlfriend. Things
    were great for a while, aaaand then they weren't. There was a bit of a messy breakup right at the end of senior year,
    roughly 3 days before graduation. Needless to say, I wasn't very focused on the graduation when it happened.

    The following summer, originally, would have been great. I was free of high school and I didn't have to do any more
    homework! Except instead, I was in this sort of depressive state. Now my grandmother has dementia and she's the kind
    of person that wanders around the whole town. She would walk on the side of the road to... well, who knows where.
    She didn't really know, no one around her knew, but she was definitely walking. In the end, someone needed to watch her
    for a while and I was the one asked to do this. Since I was just out of school, didn't know what I was doing or
    even what I wanted to do with my life, and I was cooped up in my room just playing games all day, I decided that it
    would probably be a good idea to go out and do something different. I said yes, and then I was shipped off to Maine
    for about a month.

    Things up in Maine were actually really good! The place that my grandmother lives is in a town that's basically in
    the middle of nowhere, right along the border of Canada. It's all farmland, and it's very quiet. It's the kind of 
    place where sitting outside and just watching grass grow is one of the best activities. So, by this point, I'm up
    there and I have to find some way to keep myself occupied for a month. I've got a laptop, and I've got my phone,
    and I've maybe got a couple of books. I'll figure it out.

    The issue with this whole month, in the end, wasn't that I went up to Maine. It wasn't that I was quite possibly
    bored out of my mind. The issue is that while I was up there, I didn't have anyone that was telling me to go to
    church. And, as someone that wasn't nearly as motivated about it as I am today, that meant that I wasn't going to
    church at all. I'm sure that there are people here that have skipped church before, and we all know that things
    happen to pop up that try to drag our attention away from it. That's exactly what happened to me, and I haven't
    told anyone about it up until this point. It's the only time that I can think of that I've ever missed church,
    and since then I've never done it again.

    The truth is, for the two or three weeks that I didn't go to church, I felt horrible. Worse than I thought I could, or even should
    have. It started to literally feel like some sort of weight on my shoulders, and it was constant. Even though I
    had tons of time to think about it, I continued to just fill my time with doing anything except that. I tried to
    write it off as the after-effects of the breakup that happened two months before that point. Of course, that's
    not really what it was, but I let myself believe it.
    
    During those few weeks, not only did I feel miserable, but weird things started happening. My grandmother started
    wandering more, and she started wandering at times that were completely ridiculous. One night, she decided that
    she was going to take a walk. This would have been fine, except for the fact that it was at 2am, and that the
    people that found her wandering were the cops. On another occasion, and I'm far from proud of this, I ended up
    yelling at her because of the fact that she kept wandering. It's very rare that I ever truly yell at anyone,
    and this one felt even worse because it was like yelling at a 3 year old - they're not going to get anything out
    of that.

    Finally, at the end of my month-long retreat(?) my family came up and brought me back home. Of course, there were
    other things that happened like placing my grandmother in a memory care facility and such. However, I was really 
    just happy that I was back home.

    The next week, my family and I all go to church as a family. Upon stepping into the building, I realized exactly
    what I was missing for all those week. I payed attention during that mass, and although I can't tell you which week
    it was, the readings during it, or who the priest was, I remember something clicking. I went up to recieve communion,
    and \textit{immediately} that weight from the last couple weeks was gone. I was quite literally free to be myself
    again. Since that day, I haven't missed a single weekend of mass, and I've actually tried to get a friend or two
    to come back into it.

    As I mentioned earlier, really right at the very beginning, the Eucharist is a huge part of what makes us all here
    into Catholics. This is something I learned far later into my own religious ed classes, which I think is an important
    point. We are Catholics, and basically (at least the way I've thought of it) every other Christian religion out there
    is considered Protestant. When the tally is done completely, there is a rough total of about 200 Protestant faiths
    out there. Another huge thing about these Protestant faiths, besides the fact that they're all technically branches
    of Catholicism, is that they do not worship the Eucharist. I thought this was really strange, that all of these extra
    branches of the same religion decided that they were not going to acknowledge one of the most obvious things that
    Jesus said during Holy Thursday, and it's something that we hear again ourselves every mass:
    \begin{quote}
      ...Jesus took bread, gave thanks, and broke it, and give it to his disciples, saying, ``Take and eat; \textbf{this is my body.}"
      Then he took the cup, gave thanks and offered it to them, saying, ``Drink from it, all of you. \textbf{This is my blood} of the
      covenant, which is poured out for the forgiveness of sins."
    \end{quote}
    I emphasized the parts that are really important: ``This is my body", ``this is my blood." It's also important to notice
    that even in all 4 recounts of this event, not a single one of the disciples (Mark, Matthew, Luke, and John) wrote
    anything different. This one specifically is from Matthew, but the wording is almost the same, and explicitly mention
    that he said that this \textit{is} his blood. It's not a symbol of it, it's not a piece of bread that we eat that is 
    supposed to simply remind us of what might have happened, none of that. This is truly the body and blood of Christ.

    Now, back to what I said about myself earlier. We've sectioned out an entire part of the mass for this, we've read,
    straight out of sacred scripture, that Jesus himself said that this is his body and his blood. I mean, Jesus himself.
    That's not just anyone. The issue becomes that it's still incredibly hard, at least for me, to believe in this thing.
    I mean come on! It still looks like a piece of bread. It doesn't even necessarily taste good.

    This is exactly what the Apostles said too. In fact, there were many many more of them when he said this, and it
    prompted lots of them to leave, and abandon that search for truth. John 6:53 reads: 
    \begin{quote}
      Jesus said to them, ``Very truly I tell you, unless you eat the flesh of the Son of Man and drink his blood, you
      have no life in you."
    \end{quote}
    Right after this, really just mere moments, John records this:
    \begin{quote}
      On hearing [this], many of his disciples said, ``This is a hard teaching. Who can accept it?" ... From this time 
      many of his disciples turned back and no longer followed him.
    \end{quote}
    The Eucharist isn't supposed to be easy to believe in. It's not supposed to be easy to understand, and really I think
    it's fair to say that we may never truly understand how it works until we are long dead, and we meet the creator of
    the sacrament in person. However, there's one flaw that these disciples make, and people like me make too: it's a 
    simple choice. You can either stay, and try to learn, and almost definitely fail, or you can turn your back on it
    and never learn a thing from it.

    Personally, I think the best choice is to try to learn. I'm constantly trying to learn something new about whatever it
    is that I'm doing - this could be at work, for school, heck even just something new about the game I'm playing that day -
    but it is always something new. There's tons of information on the Eucharist, tons of personal anecdotes from others
    much like what we've all been sharing here. There's some really freaky Eucharistic miracles, one or two of which actually
    have to do with the consecrated host turning into real skin and blood. However, if you simply turn away and never try
    to look for these things, you will never, ever, \textit{ever} find them. You'll never be able to believe, because you
    never gave yourself the chance.

    If I'm not wrong, you were given the chance to go to Adoration yesterday. That, in itself, is worship of the Eucharist.
    In fact, it's one of my favorite times to think about anything. Absolutely anything. You can read a book, you can pray
    about something that's troubling you, anything. The important part is that you are in the real presence of God. My
    recommendation, to anyone that is here, and specifically those that never take advantage of this, is to do so. As far
    as I know, the Blessed Sacrament is always available in the front of the church even when there is no mass going on.
    Go in the room every once in a while and just sit. It doesn't have to be for long - it could be as short or as long as
    you like - but do something with it. Jesus made this sacrament for a reason, and we should use it to its fullest.

\end{document}
