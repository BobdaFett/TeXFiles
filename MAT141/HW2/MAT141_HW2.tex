%%%%%%%%%%%% Attribution %%%%%%%%%%%%
% This template was created by 
% Chuck F. Rocca at WCSU and may be
% copied and used freely for 
% non-commercial purposes.
% 10-17-2021
%%%%%%%%%%%%%%%%%%%%%%%%%%%%%%%%%%%%%

\documentclass[11pt]{article}
\usepackage{rocca-homework}
\usepackage[letterpaper, margin=.75in]{geometry}

\title{MAT 141 Homework \#2}
\author{Lucas Vas}
\date{10/16/2023}  % One class late.

\begin{document}

%%%% Format Running Header %%%%%
\markboth{\theauthor}{\thetitle}

%%%% Insert the Title Information %%%%
\maketitle

%%%% Insert the Typed Up Problems %%%%

    \begin{problem}{Chapter 5.1 Question 87}
        Since the algorithm for creating binary numbers from integers has to do with repeatedly
        dividing by two, converting to hexadecimal is similar. We can repeatedly divide by 16 and
        use the remainder to determine the hexadecimal digit. Hexadecimal uses the numbers 0-9 and
        the letters A-F. I'll use a 4-digit hex integer for this. To convert the number
        22522 to hexadecimal:
        \begin{itemize}
            \item[] $22522 \div 16 = 1407$ remainder 10, so the first digit is A.
            \item[] $1407 \div 16 = 87$ remainder 15, so the second digit is F.
            \item[] $87 \div 16 = 5$ remainder 7, so the third digit is 7.
            \item[] $5 \div 16 = 0$ remainder 5, so the fourth digit is 5.
        \end{itemize}
        In computer science, we would also be pushing these digits onto a stack (first in last out),
        so the final hexadecimal number would be 57FA. Usually this number, when written in binary,
        would look like 101011111111010. Hexadecimal is far more efficient to calcluate and write.
        In technicality, we could also do this by converting to binary first, then to hexadecimal
        (since every hex digit translates to a 4-bit pattern in binary), but that's not what the
        question was asking for.
    \end{problem}

    \begin{problem}{Chapter 5.6 Question 18a}
        The recurrance relation that was created as part of question 17 was:
        \[a_k = 3a_{k-1}+2\]
        This is for moving a tower from pole A to pole C, but we want to transfer 
        it from pole A to pole B instead. The first 3 terms look like this:
        \begin{equation*}
            \begin{split}
                b_1 & = 1 \\
                b_2 & = 2 \text{ (move the first disk to pole C)} \\
                & = + 1 \text{ (move the second disk to pole B)} \\
                & = + 1 \text{ (move the first disk back to pole B)} \\
                & = 4 \\
                b_3 & = 4 \text{ (set up 2-disk stack on pole B)} \\
                & = + 4 \text{ (move the stack to pole C)} \\
                & = + 1 \text{ (move the third disk to pole B)} \\
                & = + 4 \text{ (move the stack to pole B)} \\
                & = 13 \\
            \end{split}
        \end{equation*}
        Our recurrance relation looks like this:
        \begin{equation*}
            b_k = 3b_{k-1} + 1, b_1 = 1
       \end{equation*}
    \end{problem}

    \begin{problem}{Chapter 5.6 Question 18c}
        To show that $b_k = a_{k-1} + 1 + b_{k-1}$ is true for all $k \geq 2$:
        \begin{equation*}
            \begin{split}
                b_2 & = a_1 + 1 + b_1 \\
                & = 2 + 1 + 1 \text{ (by definition of $a_1$)} \\
                & = 4 \\
            \end{split}
        \end{equation*}
        This is the value we should have expected (and I can't see another
        way of connecting them) Therefore, by definition, $b_2 = a_1 + 1 + b_1$.
        We then try this with $b_{k+1}$:
        \begin{equation*}
            \begin{split}
                b_{k+1} & = a_k + 1 + b_k \\
                & = (3a_{k-1} + 2) + 1 + (3b_{k-1} + 1) \\
                & = 3a_{k-1} + 2 + 1 + 3b_{k-1} + 1 \\
                & = 3a_{k-1} + 3b_{k-1} + 3 + 1 \\
                & = 3(a_{k-1} + b_{k-1} + 1) + 1 \\
                & = 3b_k + 1 \checkmark \\
            \end{split}
        \end{equation*}
        As such, we've shown that $b_k = a_{k-1} + 1 + b_{k-1}$ is true for all $k \geq 2$.
    \end{problem}

    \begin{problem}{Chapter 5.6 Question 18d}
        Personally, I think this is kind of a strange question. Of course $b_k \leq 3b_{k-1} + 1$,
        because \textit{by definition} that's like saying $b_k \leq b_k$. I don't believe that a
        proof is necessary for this, but I'll try it anyway. We'll start with the base case:
        \begin{equation*}
            \begin{split}
                b_2 & \leq 3b_1 + 1 \\
                4 & \leq 3 + 1 \\
                4 & \leq 4 \checkmark \\
            \end{split}
        \end{equation*}
        We then assume that for all $k \geq 2$, $b_k \leq 3b_{k-1} + 1$. We then can use induction:
        \begin{equation*}
            \begin{split}
                b_{k+1} & = 3b_k + 1 \\
                & \leq 3(3b_{k-1} + 1) + 1 \text{ (by assumption)} \\
                & \leq 9b_{k-1} + 3 + 1 \\
                & \leq 3(3b_{k-1} + 1) + 1 \\
                & \leq 3b_k + 1 \checkmark \\
            \end{split}
        \end{equation*}
        The issue here is that I just took the assumption, plugged it in to the equation, and then pulled it
        straight back out. I don't totally understand the idea behind this?
    \end{problem}

    \begin{problem}{Chapter 5.7 Question 57}
        We're given this recurrance relation:
        \[Y_k = E + c + mY_{k-1}\]
        Using iteration on this relation looks like this:
        \begin{equation*}
            \begin{split}
                Y_2 & = E + c + mY_{2-1} \\
                & = E + c + m(E + c + mY_{1-1}) \\
                & = E + c + m(E + c + mY_0) \\
                & = E + c + Em + cm + m^2Y_0 \\
            \end{split}
        \end{equation*}
        This starts to look a lot like a geometric series, so we can simplify it:
        \[(\sum_{i=0}^{k-1} Em^i + cm^i) + m^{k}Y_0\]
        As stated in the question, this simplifies even farther:
        \[(E + c)(\frac{m^k - 1}{m - 1}) + m^{k}Y_0\]
    \end{problem}

\end{document}
