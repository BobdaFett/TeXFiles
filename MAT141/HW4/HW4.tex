\documentclass[12pt]{article}
\usepackage{rocca-homework}

\title{MAT 141 Homework 4}
\author{Lucas Vas}
\date{\today}

\begin{document}
  \maketitle

  \begin{problem}{Section 9.2 Question 28}
    These are nested for loops in a language that I don't personally know (looks like Lua?)
    but I assume the logic follows the same. The total number of times that this loop will
    iterate is found by doing $(a-b+1) * (c-d+1)$. The reason we add the ''$+1$'' is because
    the bounds are inclusive of the values of $b$ and $d$.
  \end{problem}

  \begin{problem}{Section 9.4 Question 8}
    To solve this problem, we can create a set of sets that sum to 10 using the original
    numbers that we're given. That would look like this:
    \[S = \{ \{1, 9\}, \{2, 8\}, \{3, 7\}, \{4, 6\}, \{5\} \} \]
    If we were to pick 5 values from this set, where those values are the first value in 
    every pair, then we can see that there isn't necessarily a sum of any 2 values that
    would result in 10. If we chose 6, then we would be guaranteed a sum of 10.
  \end{problem}

  \begin{problem}{Section 9.5 Question 17}
    For some reason, I just cannot visualize this problem and I don't know why that is.
    \begin{itemize}
      \item[(a)] The straight lines can be found by doing $\binom{10}{2}$ This is equivalent to
        45. This is because every line is a combination of 2 points, and there are 10 points in
        total.
      \item[(b)] The number of straight lines that do not pass through A is found similarly -
        you're simply reducing the number of points that can be chosen from. This would be
        $\binom{9}{2}$, which is equivalent to 36.
      \item[(c)] The number of triangles that are present in this figure is found by doing
        $\binom{10}{3}$, which is equivalent to 120. Triangles, by definition, have 3 points
        that are connected by straight lines, so we choose 3 points from the 10 that are
        possible.
      \item[(d)] The number of triangles that do not pass through A is found similarly to the
        previous problem - you're simply reducing the number of points that can be chosen from.
        This would be $\binom{9}{3}$, which is equivalent to 84.
    \end{itemize}
  \end{problem}
  
  \begin{problem}{Section 9.7 Question 16}
    The problem that we're supposed to prove looks very similar to the binomial theorem. I
    noticed that the first section looks like the total number of combinations that are
    possible from the addition of sets $m$ and $n$, where you choose $r$ elements, where
    $r \leq m \vee n$. This looks like the same issue that's solved with the binomial theorem,
    as your coefficient from $(m + n)^{x}$ where $m^{r} n^{s-r}$ will yield the same answer.
 
    The equation that's presented in the problem deals with iterating through the value $r$ so
    that you can choose $r$ values from $m + n$. The first iteration will choose $r$ values from
    $m$, which leaves no choices for $n$. The second iteration will choose $r-1$ values from
    $m$, which leaves a single choice for set $n$, and so on. This will eventually result in
    $r$ values being chosen from $n$, which would sum to all combinations of $\binom{m+n}{r}$.
  \end{problem}

  \begin{problem}{(BONUS) Section 9.3 Question 26}
    Using the set of all strings of \textit{a}'s, \textit{b}'s and \textit{c}'s:
    \begin{itemize}
      \item[(a)] The list of all strings of lengths 0 through 3 that don't contain \textit{aa} is:
        \begin{itemize}
          \item[$s_0$:] $\emptyset$ - the empty string.
          \item[$s_1$:] $a, b, c$
          \item[$s_2$:] $ab, ac, ba, bb, bc, ca, cb, cc$
          \item[$s_3$:] $aba, abb, abc, aca, acb, acc, bab, bac, bba, bbb, bbc, bca, bcb,$
          \item[] $bcc, cab, cac, cba, cbb, cbc, cca, ccb, ccc$
        \end{itemize}
      \item[(b)] For all $n \geq 0$, the number of strings of length $n$ that don't contain \textit{aa}
        is:
        \begin{itemize}
          \item[$s_0$:] $1$ - just the empty string.
          \item[$s_1$:] $3$
          \item[$s_2$:] $8 = 2(3) + 2(1)$
          \item[$s_3$:] $22 = 2(8) + 2(3)$
        \end{itemize}
      \item[(c)] The recurrence relation for the number of strings of length $n$ that don't contain
        \textit{aa} is:
        \[s_{n} = 2s_{n-1} + 2s_{n-2}\]
      \item[(d)] Using this relation, the number of strings that exist (without \textit{aa}) of length
        4 is:
        \begin{equation*}
          \begin{split}
            s_{4} & = 2(22) + 2(8) \\
              & = 44 + 16 \\
              & = 60
          \end{split}
        \end{equation*}
    \end{itemize}
\end{problem}

\end{document}
