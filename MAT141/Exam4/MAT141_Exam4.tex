\documentclass[12pt]{article}
\usepackage{rocca-homework}
\usepackage[letterpaper, margin=.75in]{geometry}

\title{MAT 141 Homework 4}
\author{Lucas Vas}
\date{11/20/2023}

\begin{document}
    
    \maketitle

    \begin{problem}{Problem 1}
        There are 30 multiples of 3 - we can get this number by quite literally counting them. To use combinations,
        we would need to choose the first number and the second number, however if the first number is a multiple of 3,
        then there are 4 possibilities for the second number, as opposed to 3. So we would do this:
        \[\binom{6}{1} \binom{3}{1} + \binom{3}{1} \binom{4}{1} = 30\]
    \end{problem}

    \begin{problem}{Problem 2}
        To do this question I'll assume that the letters can only ever be uppercase and the numbers are 0-9.
        Assuming that licence plates are four letters followed by three numbers:
        \begin{itemize}
            \item[(a)] The number of plates possible is all the permutations of 4 letters and 3 digits:
                \[\frac{26!}{(26-4)!} * \frac{10!}{(10-3)!} = 1,235,520,000\]
            \item[(b)] The number of plates that start with A and end with 0 is all the permutations of
                3 letters and 2 digits, since we're given the first and last characters:
                \[\frac{26!}{(26-3)!} * \frac{10!}{(10-2)!} = 26 * 25 * 24 * 10 * 9 = 1,404,000\]
            \item[(c)] The number of plates that are completely unique is the number of combinations of 26 letters
                taken 4 at a time times the number of combinations of 10 numbers taken 3 at a time:
                \[\binom{26}{1} \binom{26}{1} \binom{26}{1} \binom{26}{1} \binom{10}{1} \binom{10}{1} \binom{10}{1} = 45,697,600\]
        \end{itemize}
    \end{problem}

    \begin{problem}{Problem 3}
        Given our graph:
        \begin{itemize}
            \item[(a)] You can get there 2 + 3(4) = 14 different ways.
            \item[(b)] This is roughly the same as above - 3(4) = 12 different ways.
        \end{itemize}
    \end{problem}

    \begin{problem}{Problem 4}
        Given that a student council has 8 men and 7 women:
        \begin{itemize}
            \item[(a)] We can form a committee with 3 men and 3 women in: \[\binom{8}{3} \binom{7}{3} = 1176 \text{ ways.}\]
            \item[(b)] We can form a six person committee with at least one woman in: \[\binom{15}{5} = 3003 \text{ ways.}\]
        \end{itemize}
    \end{problem}

    \begin{problem}{Problem 5}
        In the word HULLABALOO, there are 3 L's, 2 A's, 2 O's, 1 H, 1 B, and 1 U, for 6 distinct letters in total.
        There are also 10 letters in total. So, the number of permutations of the word can be found by doing:
        \[\binom{10}{3} \binom{7}{2} \binom{5}{2} \binom{3}{1} \binom{2}{1} \binom{1}{1} = 151,200 \text{ ways.}\]
    \end{problem}

    \begin{problem}{Problem 6}
        Given that $n$ is a positive integer, then we can find the number of solutions to $1 \leq i \leq j \leq k \leq l \leq n$ by
        "reindexing" the variables. In this case, I changed it to $0 \leq i \leq j \leq k \leq l \leq n-1$. Then, we can find the number
        of solutions by doing $\binom{n-1}{4}^4$.
    \end{problem}

    \begin{problem}{Problem 7}
        Given that $n$ is a positive integer, then we can find the number of solutions to $x_1 + x_2 + x_3 = n$ by doing:
        \begin{itemize}
            \item[] $\forall x_i \geq 0 \rightarrow \binom{n+2}{n}$
            \item[] $\forall x_i \geq 1 \rightarrow \binom{(n - 3) + 2}{n - 3}$
        \end{itemize}
    \end{problem}

    \begin{problem}{Problem 8}
        Given a deck of cards, the number of cards needed to be picked to guarantee that there are 2 cards of the same suit
        is 5 - there are 4 suits, so if you pick 5 cards, you're guaranteed to have 2 of the same suit.

        The number of cards needed to be picked to have a 50\% chance of having 2 cards of the same suit, given the same deck
        of 52 cards, is 3. This is because there are 4 suits, so if you pick 3 cards, there's a 3:4 chance of having 2 of the
        same suit (I think?)
    \end{problem}

    \begin{problem}{Problem 9}
        Assuming that division by 2373 gives a repeating decimal, the number of digits in the repeating decimal is 2372.
        This is because the number of digits in the repeating decimal is the number of digits in the number that you're
        dividing by minus 1. So, the number of digits in the repeating decimal is 2372.
    \end{problem}

    \begin{problem}{Problem 10}
        In any set of 13 integers chosen from 2 - 40, there are 2 integers that share the same divisor. This is because
        there are 12 possible divisors of the numbers 2 - 40. So, if you pick 13 numbers, you're guaranteed to have 2
        numbers that share the same divisor.
    \end{problem}

    \begin{problem}{Problem 11}
        Using the binomial theorem to expand $(p - 2q)^4$ looks like this (I filtered out the choices where you would get a value of 1):
        \[(-2)\binom{4}{4}p^4 + (-2)\binom{4}{3}p^3q + (-2)\binom{4}{2}p^2q^2 + (-2)\binom{4}{1}pq^3 + (-2)\binom{4}{0}q^4\]
        This simplifies to:
        \[2p^4 - 8p^3q + 12p^2q^2 - 8pq^3 + 2q^4\]
    \end{problem}

    \begin{problem}{Problem 12}
        Using the binomial theorem to find the coefficient of $a^5b^7$ in $(a - 2b)^{12}$ looks like this:
        \[\binom{12}{5}\binom{7}{7} = \frac{12!}{5!7!} * \frac{7!}{7!0!} = \frac{12!}{5!7!} = 792\]
        We then use this number to find the coefficient of $a^5b^7$:
        \[792 * a^5b^7 * -2 = -1584a^5b^7\]
    \end{problem}
        
    \begin{problem}{Problem 13}
        The intial case of this problem is $n = 0$, which we can then show as $\binom{n}{0} = 1$. We can also show that
        for $n = 1$, $\binom{1}{0} + 2\binom{1}{1} = 3$. By assuming that these are true, we can then show that \newline
        $\forall n \geq 0 \rightarrow 3^n = \binom{n}{0} + 2\binom{n}{1} + 2^2\binom{n}{2} + ... + 2^n\binom{n}{n}$
        by doing:
        \begin{equation*}
            \begin{split}
                3^{n+1} &= 3^n * 3 \\
                &= 3^n * 3^1 \\
                &= \left(\binom{n}{0} + ... + 2^n\binom{n}{n}\right) * \left(\binom{n}{0} + 2\binom{n}{1}\right) \\
                &= \binom{n}{0}\binom{n}{0} + 2\binom{n}{0}\binom{n}{1} + 2\binom{n}{1}\binom{n}{0} \\
                &+ 4\binom{n}{1}\binom{n}{1} + ... + 2^n\binom{n}{n}\binom{n}{0} + 2^{n+1}\binom{n}{n}\binom{n}{1} \\
                &= \binom{n+1}{0} + 2\binom{n+1}{1} + 2^2\binom{n+1}{2} + ... + 2^{n+1}\binom{n+1}{n+1}
            \end{split}
        \end{equation*}
        Sorry that this is a bit messy, I'm not sure how to format it better. The + sign on line 4 doesn't quite
        line up with what I wanted to happen.
    \end{problem} 

    \begin{problem}{Problem 14 (Bonus)}
        To find the coefficient of $x^3y^2z^5$ in the expansion of $(x + y + z)^{10}$, we can use the multinomial theorem:
        \[\binom{10}{3, 2, 5} = \frac{10!}{3!2!5!} = 2520\]
        Therefore, our final answer is $2520x^3y^2z^5$.
    \end{problem}

\end{document}
