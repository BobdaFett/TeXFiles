\documentclass[12pt]{article}
\usepackage{rocca-homework}
\usepackage[letterpaper, margin=.75in]{geometry}

\author{Lucas Vas}
\title{MAT141 Practice Exam 3}
\date{10/30/2023}

\begin{document}
    \maketitle

    \begin{problem}{Problem 1}
        \begin{itemize}
            \item[(a)] The powerset of A is \{$\emptyset$, \{1\}, \{2\}, \{1, 2\}\}
            \item[(b)] The powerset of $A \cap B$ is \{$\emptyset$, \{2\}\}
            \item[(c)] The powerset of $A \times B$ is \{$\emptyset$, \{(1, 2)\}, \{(1, 3)\}, \{(2, 2)\},\\
            \{(2, 3)\}, \{(1, 2), (1, 3)\}, \{(1, 2), (2, 2)\}, \{(1, 2), (2, 3)\},\\
            \{(1, 3), (2, 2)\}, \{(1, 3), (2, 3)\}, \{(2, 2), (2, 3)\},\\
            \{(1, 2), (1, 3), (2, 2), (2, 3)\}\}
        \end{itemize}
    \end{problem}

    \begin{problem}{Problem 2}
        Given our sets, then:
        \begin{itemize}
            \item[(a)] $A \cap B = \{x \in \mathbb{R} | 1 \leq x \leq 2\}$
            \item[(b)] $(A \cup B)^c = \{x \in \mathbb{R} | (x \leq 0) \wedge (x \geq 9) \wedge (2 < x < 3)\}$
            \item[(c)] $B^c \cap C = \{x \in \mathbb{R} | 4 \leq x \ 9\}$
        \end{itemize}
    \end{problem}

    \begin{problem}{Problem 3}
        This function is not well defined because $h\left(\frac{m}{n}\right) = \frac{m}{n}$, however the function calls for $\frac{m^2}{n}$.
        $\frac{m}{n} \neq \frac{m^2}{n}$, and thus the function is not well defined. \\ For example, if we used the input $\frac{3}{4}$,
        then we would expect the output to be $\frac{3}{4}$, however the function would return $\frac{9}{4}$. 
    \end{problem}

    \begin{problem}{Problem 4}
        Given the function $l: \mathbb{S} \rightarrow \mathbb{Z}$, where $\mathbb{S}$ is the set of all binary strings, and $l$ is the length of the string, then:
        \begin{itemize}
            \item[(a)] The function is not one to one, for example: the string "01" has the same length as the string "10", however the strings are not equal.
            \item[(b)] The function is onto, since every value in the codomain will have at least one corresponding value in the domain. There is always a string of
                       size n, where n could be an infinite number.
        \end{itemize}
    \end{problem}

    \begin{problem}{Problem 5}
        If $g \circ f$ is one to one, then $g$ must also be one to one. This is because if $g$ is not one to one, then there exists
        multiple outputs of any given output of $g$ in order to achieve the same output of $g \circ f$. This would mean that $g \circ f$ is not one to one.
        Therefore, $g$ must be one to one in order for $g \circ f$ to be one to one.
    \end{problem}

    \begin{problem}{Problem 6}
        All pairs contained in R are:
        \[(3, 4), (3, 5), (3, 6), (4, 5), (4, 6), (5, 6)\]
        All the pairs contained in $R^{-1}$ are: 
        \[(4, 3), (5, 3), (6, 3), (5, 4), (6, 4), (6, 5)\] 
    \end{problem}

    \begin{problem}{Problem 7}
        The set A looks a lot like binary quadruplets. There would be 16 total elements.
        Given the relation where $sRt$ where $s$ and $t$ have the same starting letter, then the relation is reflexive,
        symmetric, and transitive. Every element in the set is related to itself, since every element starts with its own first letter.
        Every element is transitive, for example: ABBA is related to ABAB, and ABAB is related to AABB, and ABBA is related to AABB.
        Every element is symmetric, for example: ABBA is related to ABAB, and ABAB is related to ABBA.

        Due to the fact that this is an equivalence relation, it's also worth noting that there would be two equivalence classes, A and B.
    \end{problem}

    \begin{problem}{Problem 8}
        This is really a very similar question to the last one. The set $\mathbb{R} \times \mathbb{R}$ is the set of all ordered pairs of real numbers.
        The relation $(w,x)P(y,z)$ is asking for all ordered pairs where the first number is equal to the second letter. Once again, this type of relation
        would be reflexive, symmetric, and transitive. Every element is related to itself, since the first number is equal to its own first number. Every element
        is symmetric because if (a, b) and (c, d) are related, then (c, d) and (a, b) are related, since c = a. Every element is transitive because if (a, b) and (c, d)
        are related, and (c, d) and (e, f) are related, then (a, b) and (e, f) are related, since a = c = e.

        The equivalence classes within this would be the same as the first number in the ordered pairs. For example, the equivalence class [1] would be all ordered pairs
        where the first number is 1, The equivalence class [2] would be all ordered pairs where the first number is 2, and so on.
    \end{problem}
\end{document}