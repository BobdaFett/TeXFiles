\documentclass{article}
\usepackage{rocca-homework}
% \usepackage{graphicx}

\title{MAT 141 Exam 1 Corrections}
\author{Lucas Vas}
\date{October 2023}

\begin{document}

\maketitle
\clearpage

\section{Question 3}
    \paragraph{
        With the original statement being $p \wedge (\sim r \vee q)$, the negation of the statement (using DeMorgan's law) would be $\sim p \vee (r \wedge \sim q)$. Originally, I wrote $q$ rather than $\sim q$.
    }

\section{Question 4}
    \paragraph{
        Given the statement "if you are making lasagna, then you need ricotta," we need to write the converse, inverse, contrapositive, and negation of it.
    }
    \begin{itemize}
        \item \textbf{Converse:} If you need ricotta, then you are making lasagna.
        \item \textbf{Inverse:} If you don't need ricotta, then you aren't making lasagna.
        \item \textbf{Contrapositive:} If you don't need ricotta, then you aren't making lasagna.
        \item \textbf{Negation:} You are making lasagna, but you don't need ricotta.
    \end{itemize}
    \paragraph{
        In my original answer, I flipped the converse and inverse statements.
    }

\section{Question 6}
    \paragraph{
        Once again, I flipped the converse and inverse, resulting in the wrong answer again. I've omitted the modus ponens and modus tollens parts of the question, since I got them correct.
    }
    \begin{itemize}
        \item \textbf{Converse Error:} If I watch TikTok, then I won't finish my work. I won't finish my work. Therefore, I watch TikTok.
        \item \textbf{Inverse Error:} If I watch TikTok, then I won't finish my work. I don't watch TikTok. Therefore, I will finish my work.
    \end{itemize}

\section{}

\end{document}
