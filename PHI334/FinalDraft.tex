\documentclass[12pt]{article}

\usepackage[letterpaper, margin=1in]{geometry}
\usepackage{palatino}

\begin{document}

  I've always thought that the concept of nothingness is very strange. In this class, we've talked about
  nothingness quite a bit. We read the book \textit{Being and Nothingness} by Jean-Paul Sartre, which obviously
  touches upon what he thinks nothingness itself is. We also read the collection of prefaces that Kierkegaard -
  sorry, Nicholas Notabene - wrote, which he called "prefaces to nothing." Right off the bat, we understand the
  word nothing as, well, nothing. There's nothing to be talked about, nothing to describe, and nothing to be seen.
  However, I would argue that the use of the word nothing does not simply imply the existence of something, but
  nothingness is in fact something on its own.

  Sartre's concept of nothingness, in relation to \textit{Being and Nothingness}, revolves around nothingness as
  a form of freedom. I think that he spends much of his book attempting to describe nothingness as exactly that -
  nothing. He thinks of nothingness almost as having no responsibilities, which means that you have nothing that
  others are relying upon you for, nothing that can act as an external stressor, and nothing that can change or
  alter your perception of the world around you. He argues that creating a situation where there is nothingness
  allows you to become truly free. I think this is quite interesting, especially considering that the world, in
  the way it does, will never allow us to have this freedom through nothingness until we are long dead. It is,
  unless someone comes up with a truly magnificent idea, impossible to achieve what Sartre considers true freedom.

  Sartre talks about everything using what he called phenomenology, which everyone is supposed to be able to do
  and really goes quite in-depth with the concept of objectification. Anytime Sartre demonstrates his use of any
  phenomenon, he shows that you can truly objectify anything, and that we do this quite regularly. We tend to
  objectify everything, whether these things are objects, people, places, etc. We also tend to do what I would 
  call "object wrapping." This is something that I've seen used in computer science specifically, which quite 
  literally has to do with creating a "wrapper" around an object that allows us to give it extra functionality 
  or characteristics that we want. In real life, do this with things like our cars, or phones, or really any 
  type of generalization. We create an image of that person completely based around the object that we see them
  interacting with in some way. In this case, the interactable object becomes a wrapper around the person, and
  therefore allows us to more easliy categorize people and create some sort of image for them.

  Earlier, I used the term "freedom through nothingness." I think that this could be interpreted as a situation
  that uses this concept of wrapping. In this sense, I think that there are a couple ways of looking at it. The
  first is that the nothingness is the wrapper around the concept of freedom. The second is the inverse, where 
  the freedom is the wrapper around nothingness. The first thing that needs to be addressed is the fact that an
  object must wrap another object. This requires nothingness to be some sort of object. We can achieve this by
  looking at nothingness as a thing in itself. When we refer to nothingness, we have to refer to the instance
  of nothing - even though there is a void where that thing should be, we are referring to the void. We may not
  be able to see it, or lay a finger on it, but we do refer to \textit{something}. Once we do this, we can see
  that we may wrap nothingness around something, and something around nothingness. When we actually look at
  these concepts all wrapped together, you start to see that when we use freedom as a wrapper, freedom starts
  to become a stressor in itself. To achieve freedom from nothingness, we must first use freedom as a motivator
  and use our concept of nothingness (which is still an object!) 

\end{document}
