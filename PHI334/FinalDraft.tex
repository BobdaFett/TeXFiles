\documentclass[12pt]{article}

\usepackage[letterpaper, margin=1in]{geometry}
\usepackage{palatino}

\linespread{2.0}

\begin{document}

  I've always thought that the concept of nothingness is very strange. In this class, we've talked about
  nothingness quite a bit. We read the book \textit{Being and Nothingness} by Jean-Paul Sartre, which obviously
  touches upon what he thinks nothingness itself is. We also read the collection of prefaces that Kierkegaard -
  sorry, Nicholas Notabene - wrote, which he called "prefaces to nothing." Right off the bat, we understand the
  word nothing as, well, nothing. There's nothing to be talked about, nothing to describe, and nothing to be seen.
  However, I would argue that the use of the word nothing does not simply imply the existence of something, but
  nothingness is in fact something on its own.

  Sartre's concept of nothingness, in relation to \textit{Being and Nothingness}, revolves around nothingness as
  a form of freedom. I think that he spends much of his book attempting to describe nothingness as exactly that -
  nothing. He thinks of nothingness almost as having no responsibilities, which means that you have nothing that
  others are relying upon you for, nothing that can act as an external stressor, and nothing that can change or
  alter your perception of the world around you. He argues that creating a situation where there is nothingness
  allows you to become truly free. I think this is quite interesting, especially considering that the world, in
  the way it does, will never allow us to have this freedom through nothingness until we are long dead. It is,
  unless someone comes up with a truly magnificent idea, impossible to achieve what Sartre considers true freedom.

  Sartre talks about everything using what he called phenomenology, which everyone is supposed to be able to do
  and really goes quite in-depth with the concept of objectification. Anytime Sartre demonstrates his use of any
  phenomenon, he shows that you can truly objectify anything, and that we do this quite regularly. We tend to
  objectify everything, whether these things are objects, people, places, etc. We also tend to do what I would 
  call "object wrapping." This is something that I've seen used in computer science specifically, which quite 
  literally has to do with creating a "wrapper" around an object that allows us to give it extra functionality 
  or characteristics that we want. In real life, do this with things like our cars, or phones, or really any 
  type of generalization. We create an image of that person completely based around the object that we see them
  interacting with in some way. In this case, the interactable object becomes a wrapper around the person, and
  therefore allows us to more easliy categorize people and create some sort of image for them.

  I think it's important to mention that this concept of wrapping requires both pieces to be considered as objects.
  This does mean that we need to think of nothingness as an object in this sense. We also have to think of freedom
  as an object, but I think that many of us already do that. Many times, we try to achieve freedom - it's something
  that we strive for, and something that, although we can't touch it, we understand exactly what we're working
  towards. Nothingness, on the other hand, isn't something that we usually strive for. However, I'd argue that we
  tend to strive for anything \textit{except} for nothingness. We tend to be afraid of it, and we try to run from it.
  Due to this, we can think of it as an object in its own sense. You cannot run from something that
  is nothing. In fact, you run from a void, something that we might understand as similar to a black hole. We can
  point at this void - we may not be able to point at a specific object representing nothing, but we are able to
  point at a place where something is not. For example, let's say you've lost your phone and are trying to explain
  it to someone that's helping you look. You probably say something along the lines of "it was right there," and
  point to the spot that you remember your phone being at last. In other words, you point to nothing.

  When we consider nothingness as an object, that allows us to use the concept of object wrapping. We can wrap our
  concept of nothing within something else, and give it a different meaning, or at least something extra. Earlier,
  I used the term "freedom through nothingness," which is what Sartre attempts to promote. If we unwrap this a little,
  we can see that freedom is our wrapper around nothingness. This shows that freedom actually adds itself to
  the nothingness. Nothingness, in itself, is, well, nothing, or at least it's a void of something. When we add the
  concept of freedom to it, in the way that Sartre tries to with implying that freedom comes from nothing, then we
  can see that nothingness becomes freedom itself. Nothingness becomes a goal, which it wasn't before. It becomes
  something larger than it was before, simply because we added freedom to it. Likewise, the freedom gains extra
  characteristics. Simply saying "freedom" means nothing - what are you free from? How? Where? It makes no sense
  on its own. Once you give it some sort of relation to something else, it becomes more than it was. In this case,
  Sartre gives it a relation to (in essence) our own sanity, and to nothingness, in the idea that we are free to 
  reclaim our sanity, and nothingness is our path in getting there.

  We read through Kierkegaard's prefaces to nothing as part of class. I thought these were interesting, especially
  when we had to write our own. I think that most people would simply think of these papers as a collection of
  argumentative essays. They didn't have a leadup, and they didn't necessarily have to have a follow up. They 
  presented information (not always fairly) and they were all about different things. Some of them didn't even have
  anything to do with a book, such as his \textit{ad hominem} ridicule of a local priest, or any other prefaces that
  mentioned his hometown and local happenings. However, the nothingness should be wrapped again, into something a
  little different.

  Kierkegaard, according to many philosophers, could and therefore should be interpreted differently than what he
  puts directly on the paper. Many of his works mention Christianity, which is obviously a very prevalent religion
  both now and back when he was writing. Although he writes directly about this religion, and that's usually picking
  directly at pieces of the religion itself, it's able to be interpreted as something that doesn't necessarily relate
  to Christianity at all. There's something that we would call "ambiguity" within his writing, where you may read
  what he's saying literally, or abstract it into something that we can apply to more than what he was writing about.
  I thought that the idea of the papers was something that could be looked at a little closer. I think that his idea
  of nothingness as his follow up is actually this ambiguity. I believe that he wanted you to expand this into something
  that was more relatable to you.

  Kierkegaard's writing style creates a void, something where nothingness is present. Again, the nothingness is still
  an object, it's still something that we can look towards. In this case, the nothingness is also still the base object,
  we would need to wrap something around it in order to make some sense out of it. But what do we use to do this? The
  answer is quite simple - we can use just about anything.

  Due to Kierkegaard's writing style, the interpretation of the "nothingness," just like the interpretation of
  Christianity in \textit{Concluding Unscientific Postscript}, can be almost anything. If you can make the argument
  that the text can be looked at in a specific way, then you may look at it in this way. For example, we can look at
  Christianity as a metaphor for any type of "higher power," or even simply something that we do not have control over.
  The way that we look at it, even though Kierkegaard wrote the book about it, may have nothing to do with the actual
  concept of religion.
  
  When we look at his prefaces, we do the same thing. Nothingness can once again be wrapped with another concept.
  We take nothingness, with just about no meaning on its own, and we can wrap it in just about anything. To make the
  interpretation of the nothingness as a follow up into a more personal experience, we need to come up with our own
  way of looking at the nothingness. So, once again, we take the nothingness and wrap it with something else that
  relates more directly towards our own lives, such as our own religion in some cases, a specific experience in our
  lives, or even simply reading it exactly as it is, word for word.

  In conclusion, the concept of nothingness, as shown by Jean-Paul Sartre and Soren Kierkegaard, becomes less of our
  traditional understanding of nothingness and becomes more of an object on its own. We are able to see nothingness
  as something itself, as something that we must strive for or against. We can also see that nothingness really doesn't
  have meaning on its own, and must be paired with something else in order for us to make true sense of it. In essence,
  we create our own meaning out of nothingness, which allows us to look at it however we would like. This follows what
  I would consider the core idea of existentialism: we must search through the nothingness in order to find meaning in
  our own lives.

\end{document}
