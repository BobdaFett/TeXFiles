\documentclass[12pt]{article}
\usepackage[letterpaper, margin=1in]{geometry}
\usepackage{palatino}

\begin{document}

  A preface to nothing is a contradiction in itself. Creating something like a preface that doesn't
  have anything attached to it doesn't make sense, as a preface must have something that comes after
  it. However, these prefaces can have a humorous aspect to them, which generally prompts people that
  read it to think about the issue at hand in a different sort of way.

  Typically, a preface will do a couple things - it will lay out the author's reason for writing the
  book that's attached, and it will give reasons as to why the author is credible while talking about
  the topic of the book. Obviously, if your book is actually nothing, some sort of bottomless void,
  how can you be the expert on it? There's nothing to be the expert on. Not only that, but why would
  anyone \textit{ever} write about something that doesn't actually exist? It just doesn't make sense,
  and you could make the argument that this sort of thing would be (logically, of course) hilarious.

  The humor, in this case, is really that the concept of writing about \textit{nothing} is a ridiculous
  concept in itself. Creating something that's talking about nothing, typically, is something that most
  would consider as a waste of time. Hence, someone actually doing it, in the way that Kierkegaard did,
  has a way of standing out. Since we're writing a preface to nothing, it's not like someone else is
  necessarily thinking about the same nothingness, and it's hard to prove that the nothingness is actually
  your own nothingness. It creates a sort of ambiguity that is simply supposed to make you think about
  this nothingness on your own, and to create your own opinions of it.

  Because he forces you to do this, your interpretation of the nothingness that the preface is based on
  becomes something. It's not necessarily the same thing that the writer would be focused on, or even
  something that the writer was thinking about at all. The best part about the nothingness is that it's
  completely open-ended. The nothingness can be anything that you might want it to be.

  I would argue that this is true simply because of the fact that a preface to nothing is a contradiction
  in itself. It creates a statement like ``I am about a thing and the thing does not exist." Logicians can
  use this concept to create a proof that points to just about anything, and logically, it is true. I think
  this creates an interesting part about the preface - just like I said before, it can point at and/or be
  about anything that you might want it to be.

  Overall, a preface to nothing makes absolutely no sense, and that's why it could be so brilliant.
  It prompts a sort of thought that people wouldn't normally do, as I had to with this sort of writing
  assignment. 

  \clearpage

  My idea is that this assignment doesn't really make much sense. Legitimately - creating a preface
  to something that doesn't exist literally makes no sense. However, contradictions are the root of what
  many of us would call funny, which means that a preface to nothing should be at least a little funny.

  A preface does need to be about something. And in the end, I really do believe that
  a preface is about \textit{something}, at some point or time. Even if you write this sort of preface to 
  be about nothing, because there is no book that comes after it, I would simply call the preface a sort of
  argumentative essay. It's about something, even if the content doesn't look like a book. The preface
  would be a part of something more, something that we can still see, taste, touch, etc. It then becomes
  a preface on \textit{something} because it cannot be on nothing, but it can be on something that is not a 
  book.

  In the end, creating some sort of preface on ``nothing" is really creating a preface on something, but that
  something is fairly vague and can be applied anywhere you might want it to be.

\end{document}
