\documentclass[12pt]{article}
\usepackage[normalem]{ulem}
\usepackage[letterpaper, margin=.75in]{geometry}

\begin{document}
  \section{Logical Schema}
  \begin{itemize}
    \item[] \textbf{EMPLOYEE}: (\underline{UserID}, FirstName, LastName, Status, DateActive, DateInactive, Department)
    \item[] \textbf{DEVICE}: (\underline{SerialNumber}, \dashuline{UserID}, \dashuline{WarrantyTypeID}, Brand, Model, PurchaseDate, WarrantyDate, EndOfLifeDate, Status, OrderNumber, Seller, Type, Price)
    \item[] \textbf{WARRANTY}: (\underline{WarrantyTypeID}, Description, YearsParts, YearsLabor, YearsService)
    \item[] \textbf{SERVICE\_RECORD}: (\underline{TicketID}, \dashuline{UserID}, \dashuline{TechnicianID}, \dashuline{DeviceSerial}, ServiceDate, ServiceType, ServiceReason, Price)
  \end{itemize}

  \section{Data Volume Analysis}
  Number of rows in each table:
  \begin{itemize}
    \item[] \textbf{EMPLOYEE}: 50
    \item[] \textbf{DEVICE}: 50
    \item[] \textbf{WARRANTY}: 5
    \item[] \textbf{SERVICE\_RECORD}: 50
  \end{itemize}

  Size of each row in bytes:
  \begin{itemize}
    \item[] \textbf{EMPLOYE}: (4 + 32 + 32 + 8 + 3 + 3 + 32) = 114B
    \item[] \textbf{DEVICE}: (32 + 32 + 32 + 3 + 3 + 4 + 3 + 8 + 4 + 32 + 4 + 32 + 4) = 193B
    \item[] \textbf{WARRANTY}: (4 + 4 + 4 + 4 + 255) = 271B
    \item[] \textbf{SERVICE\_RECORD}: (4 + 3 + 255 + 255 + 4 + 4 + 32 + 4) = 561B
  \end{itemize} 

  Total size of each table in bytes:
  \begin{itemize}
    \item[] \textbf{EMPLOYEE}: 50 * 114B = 5700B
    \item[] \textbf{DEVICE}: 50 * 193B = 9650B
    \item[] \textbf{WARRANTY}: 5 * 271B = 1355B
    \item[] \textbf{SERVICE\_RECORD}: 50 * 561B = 28050B
  \end{itemize}

  Total size of all tables in bytes: 44755B, or 44.755KB

\end{document}
  
